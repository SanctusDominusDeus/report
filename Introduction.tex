
\section{Introduction}%\label{qmod}


For centuries the shipping industry has played a major role in moving goods around the globe. This is true today and will continue to be as international integration increases \parencite{mohanty_modelling_2021}. Even with the recent disruptions and resulting integration push back, managers cannot entirely replace the global sourcing, instead they are choosing to reduce their supply chain exposure relative to key components to their companies \parencite{mclain_auto_2021}. The dominance of the shipping industry on international trade will persist, as it is the best option to move large quantities at a reduced costs \parencite{park_role_2019}. This allows firms to reduce costs, but also makes their production output dependent of the global supply chains. These infrastructures are instable and vulnerable to shocks, as recent events, such as, the Covid-19 pandemic or Suez-canal accident have shown us. 

Theory dictates that the output of a firm’s production function depends on productivity, labour hired and the production inputs. It is on the latter that the supply chains are relevant, as they act as linking points between companies. Moving the required inputs from where they are produced to where they are needed. To study the impact of supply networks on goods production, \textcite{elliott_supply_2020} develop a model, where the inputs are conditional on other firms. They assumed that the supplier’s link strength is endogenous because it relies on each company’s decisions to investment in improving their present relationship or to build new ones. Their main findings were the following. Improving input’s reliability comes with additional costs. In addition, there are reliability spill-overs – this means that a firm’ payoff for improving their connections is smaller than the overall benefit for the supply chain, this is known as the reliability externality. The model demonstrates that in equilibrium there is always underinvestment in the supply chains connections. We aim to test if the stock market incentivises a more reliable equilibrium point, by acting as a disciplining mechanism on this market failure.

Finally, previous empirical research on supply chain links and their implications to the financial market, have showed that for supplier-customer pairs, their accounting fundamentals and stock price movement are positively correlated \parencite{cohen_economic_2008,menzly_market_2010}. Also, more recently, \textcite{ersahin_supply_2022} measure the supply chain risk reported on each firm earnings conference calls to study how companies manage and mitigate this risk. Finally, \textcite{barrot_input_2016} analysed how the inputs specificity affects the shocks propagation throughout the production networks and its effects on the company market value. All of these, focus on exploring how the supply chain characteristics impacts the stock market. While our research question flows in the contrary direction. We are testing the effects of the financial markets on the supply chain connections strength, assessing if they can discipline the reliability externality. Taking this in consideration, our work also relates to the \textcite{hsu_pollution_2020} investigation of how the stock market can help correcting the pollution externality. We share the same economic concept, but on a different topic. 