\section{Citations}\label{qmod}
The Globalization risk premium was develop by \textcite{barrot_globalization_2019} to analyse the effect of globalization on the financial markets. They used the Shipping Costs to measure the foreign competition on the domestic market.

Many of their citations follow their methodology or add new proxies for the international trade dependency. For example, using the 10-k filings to understand the firm’s exposure to the international market of inputs and outputs, and their offshoring of production \parencite{hoberg_offshoring_2019}. Or used the impact of tariffs policy in the perceived international competition \parencite{bianconi_trade_2021}, finding similar results that exposure to globalization increases the risk premium.

However, a few criticised their approach, as the industry-level data will only reflect the average engagement on the foreign market. In the same industry, each company exposure can be different, and their conclusions can be biased \parencite{bae_best_2019}. Even \parencite{barrot_import_2018} mention the limitation of shipping costs in reflecting the unobserved costs, such as, time to ship, information barriers and contract enforcement costs, transit holdings costs, inventory costs due to possible delays on deliveries, and preparation to ship costs. That if they are not correlated with the shipping costs, will add a different conclusion to their study. 

Finally, others used new dataset to predict the international trade activity around the globe, and its economic impact. They used the Automatic Identification System (AIS) that publishes in real time the location of each ship for safety purposes. Building an indicator for the world seaborne trade \parencite{cerdeiro_world_2020}. We can use their method to generate an innovative analysis of the international trade on the stock returns, going a step back on the information flow ladder.       


